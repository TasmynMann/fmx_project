\documentclass[11pt,preprint, authoryear]{elsarticle}

\usepackage{lmodern}
%%%% My spacing
\usepackage{setspace}
\setstretch{1.2}
\DeclareMathSizes{12}{14}{10}{10}

% Wrap around which gives all figures included the [H] command, or places it "here". This can be tedious to code in Rmarkdown.
\usepackage{float}
\let\origfigure\figure
\let\endorigfigure\endfigure
\renewenvironment{figure}[1][2] {
    \expandafter\origfigure\expandafter[H]
} {
    \endorigfigure
}

\let\origtable\table
\let\endorigtable\endtable
\renewenvironment{table}[1][2] {
    \expandafter\origtable\expandafter[H]
} {
    \endorigtable
}


\usepackage{ifxetex,ifluatex}
\usepackage{fixltx2e} % provides \textsubscript
\ifnum 0\ifxetex 1\fi\ifluatex 1\fi=0 % if pdftex
  \usepackage[T1]{fontenc}
  \usepackage[utf8]{inputenc}
\else % if luatex or xelatex
  \ifxetex
    \usepackage{mathspec}
    \usepackage{xltxtra,xunicode}
  \else
    \usepackage{fontspec}
  \fi
  \defaultfontfeatures{Mapping=tex-text,Scale=MatchLowercase}
  \newcommand{\euro}{€}
\fi

\usepackage{amssymb, amsmath, amsthm, amsfonts}

\def\bibsection{\section*{References}} %%% Make "References" appear before bibliography


\usepackage[round]{natbib}
\bibliographystyle{plainnat}

\usepackage{longtable}
\usepackage[margin=2.3cm,bottom=2cm,top=2.5cm, includefoot]{geometry}
\usepackage{fancyhdr}
\usepackage[bottom, hang, flushmargin]{footmisc}
\usepackage{graphicx}
\numberwithin{equation}{section}
\numberwithin{figure}{section}
\numberwithin{table}{section}
\setlength{\parindent}{0cm}
\setlength{\parskip}{1.3ex plus 0.5ex minus 0.3ex}
\usepackage{textcomp}
\renewcommand{\headrulewidth}{0.2pt}
\renewcommand{\footrulewidth}{0.3pt}

\usepackage{array}
\newcolumntype{x}[1]{>{\centering\arraybackslash\hspace{0pt}}p{#1}}

%%%%  Remove the "preprint submitted to" part. Don't worry about this either, it just looks better without it:
\makeatletter
\def\ps@pprintTitle{%
  \let\@oddhead\@empty
  \let\@evenhead\@empty
  \let\@oddfoot\@empty
  \let\@evenfoot\@oddfoot
}
\makeatother

 \def\tightlist{} % This allows for subbullets!

\usepackage{hyperref}
\hypersetup{breaklinks=true,
            bookmarks=true,
            colorlinks=true,
            citecolor=blue,
            urlcolor=blue,
            linkcolor=blue,
            pdfborder={0 0 0}}


% The following packages allow huxtable to work:
\usepackage{siunitx}
\usepackage{multirow}
\usepackage{hhline}
\usepackage{calc}
\usepackage{tabularx}
\usepackage{booktabs}
\usepackage{caption}


\urlstyle{same}  % don't use monospace font for urls
\setlength{\parindent}{0pt}
\setlength{\parskip}{6pt plus 2pt minus 1pt}
\setlength{\emergencystretch}{3em}  % prevent overfull lines
\setcounter{secnumdepth}{5}

%%% Use protect on footnotes to avoid problems with footnotes in titles
\let\rmarkdownfootnote\footnote%
\def\footnote{\protect\rmarkdownfootnote}
\IfFileExists{upquote.sty}{\usepackage{upquote}}{}

%%% Include extra packages specified by user

%%% Hard setting column skips for reports - this ensures greater consistency and control over the length settings in the document.
%% page layout
%% paragraphs
\setlength{\baselineskip}{12pt plus 0pt minus 0pt}
\setlength{\parskip}{12pt plus 0pt minus 0pt}
\setlength{\parindent}{0pt plus 0pt minus 0pt}
%% floats
\setlength{\floatsep}{12pt plus 0 pt minus 0pt}
\setlength{\textfloatsep}{20pt plus 0pt minus 0pt}
\setlength{\intextsep}{14pt plus 0pt minus 0pt}
\setlength{\dbltextfloatsep}{20pt plus 0pt minus 0pt}
\setlength{\dblfloatsep}{14pt plus 0pt minus 0pt}
%% maths
\setlength{\abovedisplayskip}{12pt plus 0pt minus 0pt}
\setlength{\belowdisplayskip}{12pt plus 0pt minus 0pt}
%% lists
\setlength{\topsep}{10pt plus 0pt minus 0pt}
\setlength{\partopsep}{3pt plus 0pt minus 0pt}
\setlength{\itemsep}{5pt plus 0pt minus 0pt}
\setlength{\labelsep}{8mm plus 0mm minus 0mm}
\setlength{\parsep}{\the\parskip}
\setlength{\listparindent}{\the\parindent}
%% verbatim
\setlength{\fboxsep}{5pt plus 0pt minus 0pt}



\begin{document}

\begin{frontmatter}  %

\title{The Impact of Changing EU-China Trade Dynamics on South Africa}

% Set to FALSE if wanting to remove title (for submission)




\author[Add1]{Tasmyn Mann}
\ead{18395139@sun.ac.za}





\address[Add1]{Stellenbosch University, Stellenbosch, South Africa}


\begin{abstract}
\small{
Over the last century, the world has become increasingly interconnected,
as globalisation has led to the expansion of the traditional market
place. The result of this interdependence is that domestic economies are
increasingly affected, both directly and indirectly, by changes in the
economic position of foreign nations. This paper set out to determine
which areas of the SA economy are likely to feel the greatest knock on
affects from changes in trade dynamics between the European Union and
China. The research paper finds that South Africa, like many emerging
economies, is most vulnerable to external shock transmission through the
commodity price channel. As commodities make up the largest proportion
of SA exports. Further, SA is likely to impacted by any slowdown in
China's growth, as it is SA's most significant trading partner. Other
transmission mechanisms include direct trade flows, direct financial
flows, foreign direct investment and exchange rate adjusment.
}
\end{abstract}

\vspace{1cm}

\begin{keyword}
\footnotesize{
Trade Dynamics \sep EU \sep China \sep South Africa \sep Spillovers \\ \vspace{0.3cm}
\textit{JEL classification} 
}
\end{keyword}
\vspace{0.5cm}
\end{frontmatter}



%________________________
% Header and Footers
%%%%%%%%%%%%%%%%%%%%%%%%%%%%%%%%%
\pagestyle{fancy}
\chead{}
\rhead{}
\lfoot{}
\rfoot{\footnotesize Page \thepage\textbackslash{}}
\lhead{}
%\rfoot{\footnotesize Page \thepage\ } % "e.g. Page 2"
\cfoot{}

%\setlength\headheight{30pt}
%%%%%%%%%%%%%%%%%%%%%%%%%%%%%%%%%
%________________________

\headsep 35pt % So that header does not go over title




\hypertarget{introduction}{%
\section{\texorpdfstring{Introduction
\label{Introduction}}{Introduction }}\label{introduction}}

Over the last century, the world has become increasingly interconnected,
as globalisation has led to the expansion of the traditional market
place, and the transformation of macroeconomic environments. The result
of this interdependence is that domestic economies are increasingly
affected, both directly and indirectly, by changes in the economic
position of foreign nations. For emerging economies such as South Africa
(SA), this affect is more evident as changes in SA's largest trading
partners such as BRICS or the European Union (EU), have a greater long
run impact on the economy than domestic decision makers. As a primary
commodity net exporter, SA is particularly susceptible to shocks and
spill overs from these nations. Therefore, in a rapidly globalising
society, in order to ascertain future economic prospects of nation, it
becomes increasingly important to engage with the channels and
mechanisms through which SA will be affected by external developments.
This paper focuses on the evolution of trade within South Africa and
between its two large trading partners, the EU and China. Section 2
introduces the concept of globalisation and the importance of
understanding interlinkages between economies. Section 3 provides an
overview of South Africa's current economic position and details of how
current trade dynamics have evolved. Section 4 looks at SA's largest
trading partners as well as its export composition, to provide insight
on where SA is exposed to risks from foreign developements. Section 5
and 6 introduce the external environment and focus on the current trade
position of China and the EU, respectively. Section 7 focuses on the
trade dynamics between China and South Africa. Section 8 provides
insights into the channels and mechanisms through which unexpected
shocks in foreign economies may translate through to SA, specifically,
focus is placed on direct trade and financial flows, the commodity price
channel and the exchange rate adjustment channel. Section 9 concludes.

\hypertarget{globalisation}{%
\section{\texorpdfstring{Globalisation
\label{Globalisation}}{Globalisation }}\label{globalisation}}

Bhagwati (2010), defines economic globalisation as the integration of
national economies into the international market through trade.
Globalisation essentially describes the expansion of the global market
place, and the increase in trade, capital, and human capital flows, as
well as investments, across national borders. While the definition of
globalisation may be disputed, the evidence of increasing
interdependence cannot (Kastelle et al., 2006). For emerging markets,
the economic positions of their largest trading partners frequently have
a greater impact on their economies than domestic decision makers. It is
important to understand the direct and indirect impact of these changes,
as well as the magnitude of their affect. The Global Financial Crisis
(GFC) of 2008/9 exposed some of the negative consequences of
interconnected markets, as a global recession evolved from an American
mortgage crisis. The ripple effect of the recession took place through
multiple channels affecting each economy in a unique way. The GFC was
followed by a sudden and unprecedented contraction of world trade
(Baldwin, 2009). This contraction can be explained partially by a
reduction in the supply of trade finance, as developed nations struggled
to find the liquidity to sustain previous levels of trade (Freddy et
al., 2019). This liquidity crunch particularly impacted emerging
markets, and lead to sudden stops in investment and large reductions in
exports. In spite of this episode, over the last 50 years, the majority
of emerging markets have seen significant increases in manufacturing
export trade, and foreign direct investment (FDI), as a proportion of
their GDP (Kastelle et al., 2006). In addition, there has been increased
integration of capital labour markets across borders (Ghemawat, 2003).
These represent a few of the multitude of channels which expose domestic
economies to changes in foreign economic positions. All of these
channels have spill over affects which means changes in capital flows
and trade dynamics affect a variety of macroeconomic factors such as
exchange rate volatility and inflation. These examples of economic
interlinkages all represent channels through which foreign economic
activity may help or harm a domestic economy. However, the multitude of
both complementary and opposing forces, in either direction, make this
discussion incredibly nuanced when analysing even a simple bilateral
relationship. Therefore, the for the purpose of this paper, focus will
be placed on the impact of changes in trade dynamics between SA and its
largest trading partners. While the GFC literature presents examples of
how economies impact one another in the short term, this paper focuses
on how south Africa has been, and may be, impacted in the long run.

\hypertarget{south-africas-current-economic-position}{%
\section{\texorpdfstring{South Africa's Current Economic
Position\label{SAcurrent}}{South Africa's Current Economic Position}}\label{south-africas-current-economic-position}}

International trade is a major driver of growth in all countries. This
concept, referred to as the export-led growth hypothesises, has been
tested extensively in previous literature which finds that in SA,
exports have a strong and significant positive relationship with real
growth (Jung and Marshall, 2015). Over the last 5 years SA has
experienced depressed growth rates, with GDP per capita growth averaging
around 0\% since 2014 (World Bank, 2019). GDP PC growth has stagnated
due to population growth rates, and poor economic activity. This
stagnation can be seen in Figure \ref{Graph1}.

\begin{figure}[H]

{\centering \includegraphics{tradedynamics1_files/figure-latex/Graph1-1} 

}

\caption{South Africa - GDP PC 1960-2019 in Constant 2010 US Dollars \label{Graph1}}\label{fig:Graph1}
\end{figure}

Since the 1960's South Africa's real economic growth has not exceeded
6\% (Edwards and Lawrence, 2008; Freddy et al., 2019). In 2019, GDP
growth in SA was estimated to be a mere 0.4\%, which is a 4 percentage
point decrease from 2018 (World Bank, 2020). GDP growth is expected to
rise to 0.9\% in 2020, and thereafter to 1.3\%, in 2021 (World Bank,
2020). However this estimated increase is conditional on policy makers
following through with planned structural reforms, therefore reducing
policy uncertainty, as well as, the recovery of public and private
investment. The World Bank, (2020) predicts that domestic growth will be
slowed by infrastructure constraints in electricity supply as well as by
weakened external demand for SA exports (World Bank, 2020). In 2018
South Africa's GDP, in current US dollar prices, was roughly 366 Billion
(StatsSA, 2019).

Export expansion, import substitution and international capital
movements have all been important in the development of SA economy (Mohr
and Fourie, 2008). International trade accounted for about 60\% of SA
GDP, indicating that SA has a very open economy and that trade remains
incredibly important for its growth prospects (Santander, 2018). As of
2017, it was the 34th largest export economy in the world (Simoes,
2018). SARS reported that in December of 2019, total exports reached 7.2
Billion dollars while imports were roughly 6.1 Billion dollars, making
it a net exporter that month (SARS, 2020). However, Edwards and Lawrence
(2008), report that since the 1960's, import volumes have been steadily
increasing compared to export volumes. As, SA fluctuates between being a
net importer or exporter of goods, changes in trade dynamics in both
directions are equally important (CEIC, 2020).

\begin{figure}[H]

{\centering \includegraphics{tradedynamics1_files/figure-latex/Graph2-1} 

}

\caption{South Africa Exports and Imports  \label{Graph2}}\label{fig:Graph2}
\end{figure}

Figure \ref{Graph2} indicates the evolution of SA exports and imports
since 1960. Trade liberalisation in the 1990's dramatically increased
trade in both directions for SA. Since 2009, SA has seen slow export
recovery and moderate import recovery. Figure \ref{Graph10} indicates
that SA fluctuates between a net exporter and net importer. The Figure
further indicates that SA's trade position is incredibly volatile and
contingent on external environments.

\begin{figure}[H]

{\centering \includegraphics{tradedynamics1_files/figure-latex/Graph10-1} 

}

\caption{South Africa's Net Trade in Goods and Services \label{Graph10}}\label{fig:Graph10}
\end{figure}

Over 1970 and 1991, real GDP increased by 54\% while the real exchange
rate was a similar level in both years. Import volumes over this period
were relatively flat, possibly due to protectionism and political
sanctions. Non commodity exports were also relatively flat, while non
gold commodity exports increased steadily. Over this period, world trade
increase by 90\% and world GDP by 70\%, indicating that comparatively,
SA did not benefit from globalisation over this time (Edwards and
Lawrence, 2008). Between 1991 and 2001, import volumes grew by 73\% and
export volumes by 70\% respectively. Reductions in gold exports volumes
(by roughly 30\%) were offset by the 50\% increase in exports of other
commodities and a 200\% increase in non-commodity exports (Edwards and
Lawrence, 2008). Over this period exports grew by 5.7\% while
manufacturing exports grew by 7.8\% per year. Trade liberalization in
1990's played a large role in increased export revenues.

In the early 2000's, productivity growth in commodity exporting
countries as well as increases in commodity prices, lead to higher
export revenues which aided in reversing the 20\% decline in
productivity over the final decade of the Apartheid. SA benefitted from
substantial investments in commodity production and exploration, as an
oil and metal exporter (Khan et al., 2016; Schodde, 2013). During this
period, exports rose as a result of this increased productivity as well
as from an improvement in policy frameworks, increased trade openness
and foreign capital inflows resulting from the transition to democracy
(Du Plessis and Smit, 2007). Between 2000 and 2005 import volumes
increased dramatically each year, as SA sourced mostly durable goods.
Automobile imports rose at 23.5\% per year over the period and commodity
imports by 8\% per annum. In the same period, export growth in both
commodity and non-commodity goods was slow, however growth exports of
services averaged 8.6\% (Edwards and Lawrence, 2008).

Since 2008, SA economy has been slowly stagnating, achieving low levels
of growth as well as underwhelming export revenues. Contractions in the
mining and manufacturing sector have played a large role in slower
export revenue growth as the manufacturing sector decline by 8.8\% in
2019 (World Bank, 2020). Structural change is desperately needed to
diversify South Africa's revenue stream as its dependence on commodity
exports have led to increasing exposure to foreign risk.

\hypertarget{trade-partners-and-export-composition}{%
\section{\texorpdfstring{Trade Partners and Export Composition
\label{Trade Partners and Export Composition}}{Trade Partners and Export Composition }}\label{trade-partners-and-export-composition}}

South Africa's largest trading partners include China, the United State
of America, Germany, Japan, India and Saudi Arabia (Santander, 2018). In
addition, continuing negotiations around the African Continental Free
Trade Area present an avenue for potential increases in regional trade.
In terms of exports, South Africa's largest export trading partners are,
China (17.1B), the USA (8.21B), India (8B), the UK (7.97B) and Germany
(7.05B), where parenthesis indicate the value of exports in US dollars
to each nation in 2017 (Simoes, 2018). In terms of imports, China
(15.6B), Germany (7.23B), the UK (5.49B), India (4.28B) and Saudi Arabia
(3.89B) represent SA largest import trading partners (Simoes, 2018).
Clearly China is an incredibly significant trade relationship for SA.

Historically, commodity prices and external commodity demand, have had a
large impact on SA trade and growth, as it is a major exporter of
minerals and a large importer of oil. In particular, SA is still heavily
reliant on gold, diamond, platinum, cars and coal briquettes. These
items made up for 48.83 Billion dollars of exports in 2017 (Simoes,
2018). SA predominantly imports crude petroleum, refined petroleum,
cars, gold and broadcasting equipment. To the value of 21.06 billion
dollars in 2017. The composition of trade in SA, and its evolution over
time, indicate the areas in which SA is exposed to external risk and
foreign developments. A sharper than expected deceleration in major
trading partners, such as China, the EU, or the US would substantially
decrease expected export revenues and investment, as these economies
account for 40\% of the export revenue generated in Sub-Saharan Africa
(World Bank, 2020). In addition to this exposure, China accounts for one
half of global metal demand and one quarter of global oil demand,
therefore an unexpected or fast, slowdown in china's growth, would cause
a sharp fall in commodity prices. Creating a problematic environment for
SA export growth. The World Bank predicts slow GDP growth for 2020 and
2021, stating that weakening external demand particularly from the Euro
area and China, would deeply impact SA export volume, and that fiscal
revenues are dependent on these sectors (World Bank, 2020).

\begin{figure}[H]

{\centering \includegraphics{tradedynamics1_files/figure-latex/Graph3-1} 

}

\caption{South Africa Export Composition 2017 \label{Graph3}}\label{fig:Graph3}
\end{figure}
\begin{figure}[H]

{\centering \includegraphics{tradedynamics1_files/figure-latex/Graph4-1} 

}

\caption{South Africa Export Composition 1995 \label{Graph4}}\label{fig:Graph4}
\end{figure}

Figures \ref{Graph3} and \ref{Graph4}, demonstrate the change in SA's
export composition over the last 2 decades. The data is sourced from
World Trade Organisation databases and compiled by the OEC. The figures
indicate that over the last 20 years SA has managed to reduce its
dependency on mineral products and metals, however it has become more
dependent on precious metals, predominantly, gold, diamonds and
platinum, which made up for 33.3\% of all exports in 2017. The size of
SA's transortantion export market has significantly increased while the
exports of animal products, chemical products and foodstuffs has shrunk.
Figure \ref{Graph4} indicates that SA is still reliant on their
extractive sectors for exports. Overall, export composition is dominated
in primary goods, commodities or minerals. The import market is
dominated largely by manufactured goods and oil (Alence, 2015; Claar,
2018).

In addition to export cmposition, the history of SA's terms of trade
give an indication of global economic health, as the index essentially
reports an economy's international buying power. SA's terms of trade is
giving in Figure \ref{Graph5}:

\begin{figure}[H]

{\centering \includegraphics{tradedynamics1_files/figure-latex/Graph5-1} 

}

\caption{South Africa's Terms of Trade between 1960 and 2018 \label{Graph5}}\label{fig:Graph5}
\end{figure}

Terms of trade is an index of the ratio between import and export
prices. The variable gives an indication how competitive an economies
trade prices are. Where terms of trade rises, this indicates that
exports are more expensive and that imports are cheaper, which leads to
an improvement in living standards. Frequently, a deterioration in the
terms of trade index is preceeded by a devaluation in the Rand.
Appreciation in the exchange rate will improve the terms of trade index.
Since the 1960's, terms of trade in SA has been volatile, where it
steadily declined between 1980 and 2000 and then rose in the early
2000's (Edwards and Lawrence, 2008). As commodity prices fall, SA seas a
deterioration in its terms of trade, this is evident between 2011 and
2015, where comoddity prices were dropping.

\hypertarget{chinas-current-economic-position}{%
\section{\texorpdfstring{China's Current Economic Position
\label{China's Current Economic Position}}{China's Current Economic Position }}\label{chinas-current-economic-position}}

\begin{figure}[H]

{\centering \includegraphics{tradedynamics1_files/figure-latex/Graph6-1} 

}

\caption{Growth in GDP 1960-2019 \label{Graph6}}\label{fig:Graph6}
\end{figure}

Over the last few decades, China growth rates have been astounding. Fast
and dramatic growth in the East Asian nation has fundamentally changed
the global dynamics of trade (Woetzel, 2019; Paus et al., 2009). The IMF
estimated that China's GDP was roughly 14.3 Trillion dollars in 2019
(IMF, 2020). GDP growth in China has averaged around 6\% in the last 30
years. However, since the GFC China's debt to GDP ratio has been
climbing (European Comission, 2015), therefore increasing global
exposure to Chinese credit risk. Figure \ref{Graph6} provides an
indication of the major discrepancies in GDP growth rates between China,
SA and the EU. Export and Import Growth have also significantly grown
over the last 50 years.

\begin{figure}[H]

{\centering \includegraphics{tradedynamics1_files/figure-latex/Graph7-1} 

}

\caption{China Export and Import Growth \label{Graph7}}\label{fig:Graph7}
\end{figure}

China successfully positioned themselves as an upcoming economic giant
through favourable trade policies, that stimulated exports and
production. Chinas interventionist policies such as domestic subsidies
and exchange rate control have enabled it to control its terms of trade.
In addition to government intervention China has, over the years,
negotiated favourable trade deals with most of the major economies,
gaining low tariff exporting opportunities (Paus et al., 2009). However,
incredibly high growth rates have been difficult to sustain in recent
years, given increased global trade tensions. Rising tensions between
China and the USA administration particularity hampered export growth in
2017, and slowed down GDP growth to (a still impressive) 6.1\% in 2019
(World Bank, 2019). Rising tensions broke out into a full trade war in
2018 which were followed by complicated negotiations as the US attempted
to strongarm China into changing trade terms (Woetzel, 2019).
Historically the EU has been an inadvertent beneficiary of US-China
trade tensions(Satander, 2018). However, in the wake of the trade war,
Europe capitalised on the opportunity to negotiate more favourable trade
agreements with China. The expected slowdown in China in 2020, may be
temporary, if trade deals are being negotiated for economic, rather than
political reasons. If the US-china trade war is predominantly
politically motivated to the limitation of China's rise as an economic
power, then one would expect the tensions to be long lasting and for
growth to slow over the next few years. This would have a great impact
on many emerging markets both in Africa and Asia. High export growth has
only been sustained by ever increasing demand for manufactured goods in
all regions. The USA and the EU are the largest consumers of these
goods. As demand for manufactured goods decreases from the trade war or
alternatively price of imports increases as a response to the trade
agreement, there will be slower than expected increase in imports in
China. The rise of China has greatly benefitted SA thus far, however an
expected slowdown in Chinas growth could impede SA export growth.

\hypertarget{the-european-union}{%
\section{\texorpdfstring{The European Union
\label{Europe}}{The European Union }}\label{the-european-union}}

The European Union holds many of the world's major economies such as
Germany and the United Kingdom. This free trade region creates much of
the demand for manufactured goods.

\begin{figure}[H]

{\centering \includegraphics{tradedynamics1_files/figure-latex/Graph8-1} 

}

\caption{Europe Export and Import Growth \label{Graph8}}\label{fig:Graph8}
\end{figure}

The EU is another one of SA largest trading partners. Recently SA signed
the Economic partnership agreement (EU SACD) to replace the trade
development and cooperation agreement, incentivising freer trade between
the regions (Mhaka and Jeke, 2018). South Africa's trade relations with
the Euro Area changed significantly after the Apartheid Era. Between
1990 and 1994 the abolishment of sanctions meant the EU opened its
market for industrial products, with the exception of coal and steel,
under the General System of Preferences (GSP) in September 1994 (Claar,
2018). The strategic partnership developed after the cold war when there
was incentives to open trade (Masters and Hiero, 2017). In the early
1990's trade agreements between the EU and SA were favourable to the
latter, as the new regime negotiated both developmental and cooperation
elements as well as a free trade zone. The negotiations resulted in
regulated duty and tax free exports in the industry and agricultural
sector. 95\% of goods could be exported to the EU tariff free, while the
EU could export 86\% of goods tariff free (Grant, 2006). In addition, SA
negotiated for technical support in policy implementation, as well as
access to cheap loans and support of financial structures (Meyn, 2003).
While the imbalance gave SA a significant advantage, overall the
agreement stimulated trade between the regions and significantly opened
the markets. As of 2007, SA joined the EU-SADC EPA group which is a
multilateral trade agreement between the Euro area and Southern African
economies (Weismann, 2005). The agreement was signed in 2016 to
negotiate regional integration (Claar, 2018). The EU also has strong
trade ties to China. Europe is a net importer from China, however there
is significant bilateral trade in each direction estimated at over 1
billion Euros per day in 2019. China is the EU's biggest source of
imports, as well as the second biggest source of exports. While EU-China
trade relations have been changing, as Europe took a more offensive
stand in negotiations in 2019. China remains the EU largest supplier of
industrial and consumer goods such as machinery and equipment, and
footwear and clothing.

\hypertarget{china-and-sa-trade-dyanmics}{%
\section{China and SA Trade
Dyanmics}\label{china-and-sa-trade-dyanmics}}

\begin{figure}[H]

{\centering \includegraphics{tradedynamics1_files/figure-latex/Graph9-1} 

}

\caption{Europe Export and Import Growth \label{Graph9}}\label{fig:Graph9}
\end{figure}

Figure \ref{Graph9} indicates how closely linked global economies have
become in trade movements. It also shows the large discrepency in the
manitude of trade between SA and China and the EU. South Africa's
dependence on China for trade is particularly evident as Chinas net
trade appears to be a lead to SA's. The same could possibly be argued of
Europe's and Chinas relationships except for certain periods such as
2005 and 2010, where China and the EU's trade move in opposite
directions and where clearly SA net trade movement is predominantly
affected by Chinas. Since 2008 net trade in SA has been declining,
frequently facing a trade deficit which is financed either through
borrowing or savings, to stabilise the economy. As SA trade so closely
mimics that of Chinas, one can expect any adjustments in Chinas economic
position to also affect SA similarly.

The trade relationship between China and SA has been developing since
the establishment of the Peoples Republic of China in 1998. Since this
time, numerous trade agreements have been negotiated that have
established a strong bilateral trade agreement between the countries as
well as opened trade channels between informal groups such as BRICS
(Grimm et al., 2014). A strategic partnership was declared in 2004 and a
programme for a deepening strategic partnership was established in June
2006 (PMG 2010, in Grimm et al.~2014). The formation of BRICS gives SA a
significant position in continental and even global affairs (Alden and
Yushan, 2014; Mhaka and Jeke, 2018). BRICS is an intergovernmental
organization including, Brazil, Russia, India, China and finally South
Africa after its integration in December 2010. Referring to the GDP
ranking of countries according to the IMF in 2018, China, the world's
leading economic power with USD 25 270.07 billion, is at the top of the
group, followed respectively by India (USD 10 505.29 billion), Russia
(USD 4 213.40 billion), Brazil (USD 3 365.34 billion) and South Africa
(USD 789.42 billion). In the case of South Africa, the importance of its
BRICS membership goes beyond the economic aspect and is rather perceived
as ``the integration of an association of countries sharing the same
values attached to their independence and wishing to reform the global
decision-making structure'' (Shubin, 2015 in Freddy et al., 2019).

The trade relationship between SA and East Asian countries has always
been strong. In 1990 17\% of SA imports stemmed from East Asian
countries such as China, Japan and Taiwan. This number increased to 20\%
in 2009 and has been steadily growing ever since (Mhaka and Jeke, 2018).
Exports to East Asia over the same time period grew from 13\% to 21\%.
However, the sudden and dramatic economic growth in China, as well as
the strong political and economic ties it has to SA, has culminated in
China being regarded as SA's most important trading partner (DTI, 2010;
Mhaka and Jeke, 2018). The substantial value of exports and imports
exchanged between SA and China suggests that SA has become increasingly
exposed to changes in the Chinese economy. Since 2009 China has been
SA's largest trading partner but the trade balance is in favour of the
former (Ensor, 2014; Mhaka and Jeke, 2018). The unique partnership
operates at a bilateral, continental and multilateral level.

Since the end of the global financial crisis the trade flows between
China and South Africa have been increasing (Mhaka and Jeke, 2018).
Exports to China have increased from about R2.7 billion in 2009 to R11.8
billion in November 2019 (DTI, 2020). Roughly, 90\% of SA's top 10
exports to China are comprised of raw materials. These exports are
primarily commodities such as ores, slag and ash, iron and steel, wood
pulp, gems precious metals and coins, wool, oil, copper, plastics,
machines engines and pumps, and nickel. SA has an abundance of mineral
resources (Alden and Yushan, 2014), which China imports as inputs for
their manufacturing industries. The output from these industries is sold
back to SA. This bilateral agreement is strengthened by comparative
advantage that forms the basis of their trade. In many years, SA's
export performance to China have been fluctuating but remained positive
except during 2008--2010 as shown by Mhaka and Jeke (2018). The majority
of SA imports from China stem from the secondary sector where China has
a comparative advantage. Imports from China are predominantly electronic
equipment, machines, engines and pumps, footwear , clothing, furniture
and lighting, iron or steel products and vehicles (Mhaka and Jeke,
2018).

\hypertarget{multilateral-trade-dynamics-between-sa-china-and-the-eu}{%
\section{Multilateral Trade Dynamics between SA, China and the
EU}\label{multilateral-trade-dynamics-between-sa-china-and-the-eu}}

In a globalised society, shocks and spill overs translate among all
economies. The consequences of an interconnected world can be positive
or negative. There exists a multitude of complicated linkages between
these economies, some of which are direct, and easily measured, and
others which are not simple to quantify. Frequently, the indirect
impacts are much more substantial than the direct impacts (Kaplinsky et
al., 2010).

The main channels through which SA is exposed to changes in foreign
economies are, direct trade flows, direct financial flows, the commodity
price channel and exchange rate adjustments. In addition they may impact
SA through the environment, technology transfer and integration of value
chains or through participation in institutions of regional and global
governance (Kaplinsky et al., 2010). When analysing how impacts
translate through these channels, forecasting potential outcomes and the
magnitude of affect becomes onerous. As the spill over mechanisms are
sometimes complimentary and other times represent opposing forces. This
becomes particular difficult when analysing multilateral trade
agreements. For example, a unexpected stagnation in Chinas growth could
mean a dramatic decrease in SA's export revenues, and a significant
depreciation of the Rand as its linked closely to the movements in the
Yuan. However, if the EU responded to the new sentiment by moving demand
to Sub-Saharan Africa, then it could result in a large increase in
imports from SA, which could offset the sudden stop in Chinas demand for
SA commodities. Therefore, the indirect impacts occur as a result of
relations with third parties working their way indirectly through to SA.
As demonstrated, the discussion has a tendency to become overly
hypothetical, therefore for the purpose of this paper, focus is placed
on the channels and transmission mechanisms, rather than the prediction
of potential outcomes. Thus far SA has benefitted greatly from piggy
backing off Chinas growth, who has in turn achieved their success
through EU and US demand.

\hypertarget{trade-flows}{%
\subsection{Trade Flows}\label{trade-flows}}

Firstly, Direct impacts such as trade flows between China and SA are
clear and easily measured. In trade, the relationship is mostly mutually
beneficial as SA supplies the commodities China requires for growth and
China provides SA with cheap consumer and capital goods (Kaplinsky et
al., 2010) This being said, the relationship only remains complimentary
if imported goods in SA do not crowd out SA industry, as it strives to
move away from a primary good export position (Kaplinsky et al., 2010;
Giovarnetti and Senfilippo, 2016). However, the composition of trade has
changed over the last 10 years and SA has been exposed to greater risk
as exports to China are now comprised mainly of raw materials (Ensor,
2014; Mhaka and Jeke, 2018) This has particularly impacted SA's
manufacturing sector, which has had slow growth and high unemployment
rates since its inception (DTI, 2010; Rodrik, 2008). Rapid growth in
imports from China has been seen to contribute to the
deindustrialisation of the economy, and forced it to remain as a primary
commodity exporter (Maia, 2011). Therefore, imports from China may have
affected the development of SA's manufacturing sector and increased
imports may continue to do so. In addition, since SA is so reliant on
China for export revenues, any decrease in demand for primary goods may
negatively affect SA growth rates. Particularly, if the EU takes a
strong stance alongside the US for more favourable trade agreements,
there may be a decrease in demand for SA goods (Edwards and Jenkins,
2015). On the contrary an increase in demand for Chinas manufactured
goods from the rest of the world could greatly enhance SA growth
prospects, if it provides enough export revenue not just to service
debts but to develop other industries. This is provided that the export
revenues are high enough for SA to reduce its dependency on primary
commodities. Indirectly, trade flows may have negatively impacted labour
intensive industries in SA. However, Edwards and Jenkins (2015), find
that imports from China tended to raise productivity within industries
and contributed to lower producer price inflation therefore keeping
consumer prices lower, as a result of efficiency gains. If the EU
negotiates trade agreements that cause Chinas export prices to rise, it
may create an opportunity for SA to develop manufacturing sectors to
compete with these prices, as SA still has incredibly favourable trade
deals with the EU. This outcome may give SA the chance to increase local
production, specifically in industries of strength such as textile
production.

\hypertarget{financial-flows}{%
\subsection{Financial Flows}\label{financial-flows}}

Secondly, changes in direct financial flows will impact SA greatly. Over
40\% of daily trades in the JSE come from foreign investors, and SA is
still relatively reliant on FDI. Sudden stops in capital will reflect
negatively on the Current Account as well as in GDP growth. SA is
susceptible to sudden stops in capital, as investment downgrades
continue to threaten the stability of the economy in foreign eyes (World
Bank, 2020). The SARB does not control the outflow of capital from JSE
investments, meaning sudden stops are likely if SA is downgraded
(Satander, 2018). Dependency on foreign investment, increases SA
exposure to foreign shocks. If China struggles to maintain growth, or
the EU is greatly impacted by Brexit, then over time reduced trade and
investment would lead to the shrinking of the SA economy (Summers,
2017). China has over 20 billion dollars' worth of investments in
Africa, and signed 25 agreements with SA to the value on 16.5 billion
dollars in 2015, in the hope of creating special economic zones. In
addition, China pledged a 15 billion dollar investment to SA power
utilities and infrastructure in 2018. A loss of momentum in Chinese
investment in Africa would certainly hamper growth prospects. However,
donations from the EU could easily offset these affects. Financial flows
between China and the EU are limited by restrictions on cross-border
financial transactions, investments and banking activities in China.
Therefore slower Chinese growth is unlikely to impact the EU through the
financial flows channel (European Commission, 2015). European Banks have
relatively low direct exposure to China. However European financial
markets do react significantly to negative news about Chinas economy and
equity markets, leading to higher volatile in financial flows globally
(European Commission, 2015). Emerging markets growth prospects have
weakened leading to significant adjustments in exchange rates, credit
markets and volatility levels, as these economies are exposed to capital
flight and balance sheet exposure (World Bank, 2020). Slower Chinese
growth could trigger a rise in corporate defaults, increasing
deleveraging pressures and financial market volatility. The EU is
exposed to these markets as they make up for almost 10\% of EU foreign
claims. Therefore, if other developing nations impacted negatively by
problems in China there could be spill over effects affecting European
banks.

\hypertarget{the-commodity-price-channel}{%
\subsection{The Commodity Price
Channel}\label{the-commodity-price-channel}}

Thirdly, the commodity price channel presents a channel through which SA
is exposed to the greatest risk and also has the potential to reap the
greatest rewards. It is an important source of world demand particularly
for metals and energy commodities, which represent a large composition
of SA exports (European Commission, 2015). Indirect impacts of changes
in external demand are felt in one dimension through global prices, as
China is a large player in the price setting game. China's increased
demand for commodities would raise global prices, which could impact SA
negatively as a price taker in imports, such as oil, or positively in
exports such as precious metals. Volatile commodity prices are as a
result of changes in external demand. Between 2010 and 2015, Chinas
decrease in demand for metals, let to a 33\% decrease in iron ore prices
(European Commission, 2015). If China's economy shifts from export-led
growth to consumption-led growth, then metal commodity prices would
decrease significantly. As China's slowdown sustains, SA's weak economic
growth continues to get weaker, after the manufacturing and mining
sector were negatively affected by dropping commodity prices in 2018
(World Bank, 2020). China's trade negotiations with SA indirectly impact
its exposure to African trade agreements as SA is an influential economy
in Africa. Overall, the nations have successfully realised the
comprehensive strategic partnership envisaged in 2010 (Alden and Wu,
2014; Mhaka and Jeke, 2018). Lower commodity prices may lead to downward
adjustments in the Rand as global capital flows adjust quickly to the
changing environment. There is also potential for more abrupt
adjustments in asset prices and global risk premia, if global outlook
worsens significantly.

\hypertarget{exchange-rate-adjustments}{%
\subsection{Exchange Rate Adjustments}\label{exchange-rate-adjustments}}

\begin{figure}[H]

{\centering \includegraphics{tradedynamics1_files/figure-latex/Graph11-1} 

}

\caption{Real Effective Exchange Rate in China, SA and the EU \label{Graph11}}\label{fig:Graph11}
\end{figure}

Lastly, exchange rate adjustment is an obvious consequence of
interdependence. Weaker real exchange rates reduce imports and stimulate
exports which leads to inflationary pressure, and in the SA case,
usually is followed by adjustments from the SARB (Edwards and Lawrence,
2008). The South African Rand is traded on the ``carry trade'' which is
a money market where foreign exchange brokers base their valuations on a
countries economic strength and future prospects. Recently market
sentiment suggest that the Rand is too weak to invest in. Exchange Rate
adjustments which are caused partially by changes in trade demand,
therefore have a spiralling affect, as trade demand reduces, the
exchange rate devalues, investment drops and the cycle continues. That
being said Mhaka and Jeke (2018), analyse the bilateral trade agreement
between China and SA using a Gravity Model and determine that while
exchange rate volatility impacts trade negatively, it has a limited
contribution to trade changes. However, if emerging market currencies
are placed under pressure from slow Chinese growth, then there is likely
to be negative consequences for the euro. A slowdown in China could have
limited direct effect on other economies in isolation however, the
synchronisation of this effect across many smaller economies may lead to
unexpectedly large adjustments. Asian economies are particular exposed
to Chinas growth fluctuations which could affect the European financial
system through indirect channels. A sharper-than-expected slowdown in
China could therefore lead to an abrupt adjustment in global asset
markets, as investor risk aversion increases and capital flows reduce.

\hypertarget{conclusion}{%
\section{Conclusion}\label{conclusion}}

Due to the scale and structure of the trade linkage between SA and
China, any significant and negative adjustment in China's economic
position is likely to negatively impact SA growth prospects. This
exposure to China, could also serve as a benefit, as the linkages have
historically led to South Africa piggy backing off China's success.
Direct trade linkages make SA exposed export revenue fluctuations and
import price fluctuations. Trade may be greatly affected indirectly,
through trade with other emerging economies that have strong links to
China. SA is also exposed to FDI and investment fluctuations from both
China and the EU. Increased financial market volatility may arise as a
result of negative spill overs into other emerging markets if China's
slow down persists. The strongest impact, that changes in China-EU trade
relations would have on SA, is through commodity price volatility.
Commodity price volatility strongly impacts demand for SA's main export
products, however lower commodity prices are beneficial for the European
economy. If SA can reduce its dependence on commodity exports then EU
demand for goods could offset the negative impact on commodity trade.
This could also take place through the exchange rate adjustment channel
as downward pressure on the rand leads to an appreciation of the Euro
making SA exports more attractive.

\newpage

\hypertarget{bibliography}{%
\section{Bibliography}\label{bibliography}}

Alden, C. and Wu, Y., 2014. South Africa-China Relations: Evolving
Cooperation, Collaboration and Competition.

Alence, R., 2015. Democratic Trajectories in Africa: Unravelling the
Impact of Foreign Aid. Edited by Danielle Resnick and Nicolas van de
Walle. WIDER Studies in Development Economics. Oxford University Press,
Oxford. 2013. xvi+ 310 pp.~Hbk£ 60.00. Economica, 82(327), pp.587-588.

Baldwin, R.E. ed., 2009. The great trade collapse: Causes, consequences
and prospects. Cepr.

Bhagwati, J., 2010. Banned aid: Why international assistance does not
alleviate poverty.

CEIC Data. 2020. South Africa Total Exports {[}2004 - 2020{]} {[}Data \&
Charts{]}. {[}online{]} Available at:
\url{https://www.ceicdata.com/en/indicator/south-africa/total-exports}
{[}Accessed 4 Feb.~2020{]}. Claar, S., 2018. International Trade Policy
and Class Dynamics in South Africa. Editor Timothy M. Shaw

DTI, 2010, Industrial policy action plan: Economic sectors and
employment cluster, IPAP, South Africa, pp.~43-59.

Du Plessis, S. and Smit, B., 2007. South Africa's growth revival after
1994. Journal of African economies, 16(5), pp.668-704.

Dube, S.S., 2014. The impact of exchange rate volatility on
international trade between South Africa, China and USA: The case of the
manufacturing sector (Doctoral dissertation, University of
Johannesburg).

Edwards, L. and Jenkins, R., 2015. The impact of Chinese import
penetration on the South African manufacturing sector. The Journal of
Development Studies, 51(4), pp.447-463.

Edwards, L. and Jenkins, R.O., 2014. The competitive effects of China on
the South African manufacturing sector. Development Policy Research
Unit, University of Cape Town.

Edwards, L. and Lawrence, R., 2008. South African trade policy matters
Trade performance and trade policy. Economics of Transition, 16(4),
pp.585-608.

Ensor, L., 2014, Bilateral trade with China on the increase, viewed 21
June 2015, from

European Commission. 2015. European Economic Forecast. Luxembourg:
Publications Office of the European Union, pp.53-57.

Freddy, V., Bidoley, A., Sylvestre, A.Y., Aleksandrovich, S.S. and
Viktorovna, S.N., 2019. Brazil and South Africa: The Weak Links of the
BRICS. J. Pol. \& L., 12, p.15.

Gani, A., 2017. The logistics performance effect in international trade.
The Asian Journal of Shipping and Logistics, 33(4), pp.279-288.

Ghemawat, P., 2003. The forgotten strategy. Harvard business review,
81(11), pp.76-84.

Giovannetti, G. and Sanfilippo, M., 2016. Do Chinese exports crowd-out
African goods? An econometric analysis by country and sector. In The
Power of the Chinese Dragon (pp.~10-41). Palgrave Macmillan, London.

Grant, J.H. and Meilke, K.D., 2006. The World Trade Organization special
safeguard mechanism: a case study of wheat. Review of Agricultural
Economics, 28(1), pp.24-47.

Grimm, S., 2014. China--Africa Cooperation: promises, practice and
prospects. Journal of Contemporary China, 23(90), pp.993-1011.

Grimm, S., Kim, Y., Anthony, R., Attwell, R. \& Xioo, X., 2014, South
African relations with China and Taiwan: Economic realism and the
one-China doctrine, viewed 23 June 2015, from

Jenkins, R. and Edwards, C., 2006. The economic impacts of China and
India on sub-Saharan Africa: Trends and prospects. Journal of Asian
Economics, 17(2), pp.207-225.

Jenkins, R., Peters, E.D. and Moreira, M.M., 2008. The impact of China
on Latin America and the Caribbean. World Development, 36(2),
pp.235-253.

Jung, W.S. and Marshall, P.J., 1985. Exports, growth and causality in
developing countries. Journal of development economics, 18(1), pp.1-12.

Kaplinsky, R. and Messner, D., 2008. Introduction: The impact of Asian
drivers on the developing world. World Development, 36(2), pp.197-209.

Kaplinsky, R., McCormick, D. and Morris, M., 2007. The impact of China
on sub-Saharan Africa.

Kaplinsky, R., McCormick, D. and Morris, M., 2010. China and Sub Saharan
Africa: impacts and challenges of a growing relationship.

Kastelle, T., Steen, J. and Liesch, P., 2006, June. Measuring
globalisation: an evolutionary economic approach to tracking the
evolution of international trade. In DRUID Summer Conference on
Knowledge, Innovation and Competitiveness: Dynamics of Firms, Networks,
Regions and Institutions-Copenhagen, Denmark, June (pp.~18-20).

Maia, J., Giordano, T., Kelder, N., Bardien, G., Bodibe, M., Du Plooy,
P., Jafta, X., Jarvis, D., Kruger-Cloete, E., Kuhn, G. and Lepelle, R.,
2011. Green jobs: An estimate of the direct employment potential of a
greening South African economy. Industrial Development Corporation,
Development Bank of Southern Africa, Trade and Industrial Policy
Strategies.

Masters, L. and Hierro, L., 2017. Ten years of the EU--South Africa
Strategic Partnership. South African Journal of International Affairs,
24(2), pp.113-114.

Meyn, M., 2003. The TDCA and the Proposed SACU-USA FTA: Are Free Trade
Agreements with Industrialised Countries Beneficial for SACU?. Nepru.

Mhaka, S. and Jeke, L., 2018. An evaluation of the trade relationships
between South Africa and China: An empirical review 1995-2014. South
African Journal of Economic and Management Sciences, 21(1), pp.1-15.

Mohr, P.F. and Fourie, L., L. and Associates.(2008). Economics for South
African students, 4th ed. Van Schaik: Pretoria.

Muteba Mwamba, J. and Dube, S., 2014. The impact of exchange rate
volatility on international trade between South Africa, China and USA:
The case of the manufacturing sector.

Osei-Hwedie, B.Z., 2012. The dynamics of China-Africa cooperation. Afro
Asian Journal of Social Sciences, 3(3.1), pp.1-25.

Paus, E., Prime, P. and Western, J., 2009. Global giant: is China
changing the rules of the game?. Springer.

Rodrik, D., 2008. One economics, many recipes: globalization,
institutions, and economic growth. Princeton University Press.

Santandar Trade Markets. (2018). South Africa Foreign Trade in Figures.
{[}online{]} Available at:
\url{https://santandertrade.com/en/portal/analyse-markets/south-africa/foreign-trade-in-figures?\&actualiser_id_banque=oui\&id_banque=0\&memoriser_choix=memoriser}
{[}Accessed 3 Feb.~2020{]}.

Schodde, R.C., 2013. The impact of commodity prices and other factors on
the level of exploration. Centre for Exploration Targeting, University
of Western Australia: Perth, Australia.

Shambaugh, D.L., 2013. China goes global: The partial power (Vol. 409).
Oxford: Oxford University Press.

Simoes, A. 2020. OEC - South Africa (ZAF) Exports, Imports, and Trade
Partners. {[}online{]} OEC. Available at:
\url{https://oec.world/en/profile/country/zaf/\#Exports} {[}Accessed 1
Feb.~2020{]}. Statssa.gov.za. (2020). GDP in the second quarter of 2019
increased by 3,1\% \textbar{} Statistics South Africa. {[}online{]}
Available at: \url{http://www.statssa.gov.za/?p=12482} {[}Accessed 4
Feb.~2020{]}.

Summers, T., 2017. Brexit: Implications for EU--China Relations. Chatham
House: Royal Institute of International Affairs.

Weisman, D.L., 2005. Assessing Market Power: The Trade-off Between
Market Concentration and Multi-Market Participation. Journal of
Competition Law and Economics, 1(2), pp.339-354.

Woetzel, J., 2019. China and the world: Inside the dynamics of a
changing relationship.

World Bank. 2020. Overview. {[}online{]} Available at:
\url{https://www.worldbank.org/en/country/southafrica/overview\#1}
{[}Accessed 3 Feb.~2020{]}.

World Bank. 2020. Global Economic Prospects, January 2020 : Slow Growth,
Policy Challenges. Washington, DC: World Bank. © World Bank.
\url{https://openknowledge.worldbank.org/handle/10986/33044} License: CC
BY 3.0 IGO.

Zafar, A., 2007. The growing relationship between China and Sub-Saharan
Africa: Macroeconomic, trade, investment, and aid links. The World Bank
Research Observer, 22(1), pp.103-130.

Zaman, K., bin Abdullah, A., Khan, A., bin Mohd Nasir, M.R., Hamzah,
T.A.A.T. and Hussain, S., 2016. Dynamic linkages among energy
consumption, environment, health and wealth in BRICS countries: green
growth key to sustainable development. Renewable and Sustainable Energy
Reviews, 56, pp.1263-1271.

% Force include bibliography in my chosen format:

\bibliographystyle{Tex/Texevier}
\bibliography{Tex/ref}





\end{document}
